\chapter{Instrument Response Functions and Sensitivity of FACT}\label{chp:performance}

In this chapter, I will present the sensitivity and other performance characteristics of
\gls{fact} together with the analysis methods described in Chapters \ref{chp:preprocessing} and
\ref{chp:reconstruction}.

First, I will present the used datasets and an event pre-selection in \autoref{sec:datasets},
then I will discuss the instrument response functions in \autoref{sec:irf}.
In \autoref{sec:sensitivity} the flux sensitivity will be evaluated on observed and simulated
data.

\section{Datasets}\label{sec:datasets}

In the following, I will present the simulated and observed datasets
used for estimating the sensitivity of \gls{fact} as well as the
event pre-selection.
Finally, I will look into the agreement between simulated and observed data.


\subsection{Simulated datasets}

Four simulated datasets are used for the analysis:
\begin{itemize}
  \item Diffuse protons are needed for the training of the particle classification 
    and the performance evaluation.
  \item Diffuse gamma rays are needed for the training of all four models.
  \item Gamma rays from a point source observed in wobble mode are needed
    for the calculation of the instrument response functions and the sensitivity calculation.
  \item Diffuse helium nuclei are simulated as additional background class but only 
    used for comparison of observations and simulations.
\end{itemize}
The diffuse and point-source gamma rays are generated from the same \gls{corsika} 
simulations through different settings for \texttt{CERES}.
The run-wise settings for the corsika simulations are listed in \autoref{tab:corsika-runs}.
For all simulated datasets, the azimuth angle was simulated over the whole
\ang{360} range, each individual run covering \ang{10}.
The zenith angle was simulated between \ang{0} and \ang{30}, each run with a
range of \ang{1}.

\begin{table}
  \centering
  \caption{Overview of the \gls{corsika} runs performed for this work}
  \label{tab:corsika-runs}
  \begin{tabular}{
      l
      S[table-format=1.1]
      S[table-format=3.0]
      S[table-format=+1.1]
      S[table-format=3.0]
      S[table-format=5.0]
      S[table-format=2.0]
      S[table-format=5.0]
    }
    \toprule
    Primary &
    {$E_{\min} \mathbin{/} \si{\TeV}$} &
    {$E_{\max} \mathbin{/} \si{\TeV}$} &
    {$\gamma$} &
    {$R_{\max} \mathbin{/} \si{\meter}$} &
    {$N_\text{showers}$} &
    {Reuse} &
    {$N_\text{runs}$} \\
    \midrule
    Gamma & 0.2 & 100 & -2.0 & 300 & 5000 & 1 & 10800 \\
    Proton & 0.1 & 200 & -2.0 & 500 & 10000  & 20 & 21600 \\
    Helium & 0.1 & 200 & -2.0 & 500 & 5000  & 20 & 1080 \\
    \bottomrule
  \end{tabular}
\end{table}

The simulations were performed at the LiDO3~\cite{lido3} cluster at TU Dortmund
using the \texttt{mopro3} orchestration software described in \autoref{sec:mopro} and
would have taken approximately 75 years on a single CPU.
The output size of the \texttt{eventio} files containing the Cherenkov photons on the 
ground have a total size of \SI{6}{\tera\byte} for the gamma rays and \SI{68}{\tera\byte}
for the protons after compression using \texttt{zstd}.

The datasets simulated for \cite{phd-temme} using \gls{mmcs} 6.5 introduced
an additional photon acceptance (\param{APA}) for the detector simulation,
that only accepts photons with a probability of \SI{85}{\percent}, 
randomly discarding \SI{15}{\percent} of the photons reaching the detector.
This was motivated by unsimulated obstruction of light by the telescope's structure and
neglected imperfect transparency of the Winston cones.
However, as described in \autoref{chp:preprocessing},
the way \facttools{} handles the gain was changed.
As this also has an effect of the reconstructed amount of light,
the value for this setting is also revisited.
Three different sets of simulations were produced, using \param{APA} values of \SI{85}{\percent},
\SI{95}{\percent} and deactivating it completely (\SI{100}{\percent}).
All other settings for the detector simulation stayed unchanged from \cite[131\psq]{phd-temme}
The full analysis will be performed for the three different settings independently
and the results will be compared.

The detector simulation for the simulated gamma rays is done twice,
once to produce the diffuse gammas using a maximum scattering angle of \ang{6}
and once to produce gammas from a point source observed in wobble mode using a wobble distance of \SI{0.6}{\degree}.
In total, this leads to eight simulated datasets with an overall size of \SI{25}{\tera\byte},
which was processed by \facttools{} using the \texttt{erna} package~\cite{erna},
also on \texttt{lido3}.
\texttt{erna} allows to directly collect the output of many parallel running \facttools{}
jobs into a single \texttt{hdf5} file, as used for the further analysis.
The \facttools{} processing is by far the fastest processing step.
Compared to the 75 CPU years for the \gls{corsika} simulations, \num{5.3} CPU years
for the \texttt{CERES} simulations, running \facttools{} only took under one CPU year.

The proton dataset is split into two parts,
one for training of the separation model using a quarter of the simulated events
and the remaining three quarters for the performance evaluation.
The diffuse gamma simulations are only used for the training of the energy, separation and
\param{disp} models and the point-source gammas are only used for
the performance estimation and the \gls{irf} calculation.

If no mention of \param{APA} is made, results are shown for the dataset with $\param{APA}=\SI{85}{\percent}$ and results for the other datasets are
in Appendix~\ref{apx:apa-not-85}.

\subsection{Crab Nebula Observations}
To evaluate the quality of the simulations, calculate the sensitivity
on observed data and estimate the spectral energy distribution 
of the Crab Nebula a set of high quality runs taken between October 2013 and February
2014 with a total observation time of \SI{91}{\hour} is used.
This is the same data sample as used in \cite{phd-temme} for direct comparison.
The runs were selected using the following criteria, also from \cite{phd-temme}:
\begin{description}
  \item[Camera current] below \SI{8}{\micro\ampere}. The current created by the
    \glspl{sipm} is a direct measure for the amount of \gls{nsb} photons.
    This limits the sample to dark night conditions.
  \item[Moon zenith distance] greater than \ang{100}, excluding runs where
    the moon is above or just below the horizon.
  \item[Zenith distance] of the source smaller than \ang{30}.
    This eliminates the need for special treatment of zenith dependence and offers the best sensitivity.
  \item[Trigger rate] between 40 and 85 Events per second. This is the nominal operating
    range of \gls{fact}, higher or lower values indicate bad environmental conditions
    like clouds (lower rate) or illumination by a light source like car lights (higher rates).
  \item[Trigger readiness] during more than \SI{95}{\percent} of the observation time,
    this mainly excludes runs were the strong LASER light of the \gls{magic} LIDAR system
    interfered with data taking.
  \item[Trigger threshold] smaller than \num{350} also indicating dark nights.
\end{description}
The \gls{sql} query to extract the runs from \gls{fact}'s database is in Appendix~\ref{apx:crab-runs}.

A tool to collect the single \gls{fits} files created by the automatic processing
described in \autoref{sec:erna} provided either with a list of runs or such selection
criteria into a single \texttt{hdf5} file has been implemented into the \texttt{erna} package
and was used to create the dataset of the Crab Nebula observations.

Raw data for a subset of this dataset containing runs worth \SI{17.7}{\hour} of observation time
was published in November 2017 by the \gls{fact} collaboration \cite{fact-open-data}.
An analysis of the publicly available data based on the analysis described in this work,
using the \facttools{} and \texttt{aict-tools} has been published at \cite{open-data-analysis}.
The reconstructed event data was converted into the 
data format for open-gamma ray astronomy \cite{ogadf} by \cite{phd-bruegge} and used for the first
combined analysis of Crab Nebula observations by all currently operating Cherenkov
telescopes in \cite{open-crab}.


\subsection{Event Pre-Selection}\label{sec:precuts}
The target of the event pre-selection is to remove hard to reconstruct, not properly
or not at all simulated events.
The selection criteria are discussed in detail in \cite[56\psqq]{phd-temme} and are
now applied using the \texttt{aict-tools} to all datasets.
This pre-selection if performed before training the reconstruction models.
The pre-selection reduces the amount of events for observed data roughly by a factor of two, compare \autoref{tab:events}, and dramatically improves the performance of the classifiers.
These are the criteria:
\begin{align*}
  \param{num\_pixel\_in\_shower} &\geq 10 &
  \param{num\_islands} &< 8 \\
  \param{length} &< \SI{70}{\milli\meter} &
  \param{width} &< \SI{35}{\milli\meter} \\
  \param{leakage1} &< 0.6  &
  \param{leakage2} &< 0.85
\end{align*}


\subsection{Comparison of Observations and Simulations}

\begin{figure}
  \centering
  \includegraphics[page=6]{build/plots/data_mc_comparison.pdf}
  \caption{%
    Image parameter \param{size} for the three different datasets normalized to
    the expected event rate for one hour of observation time.
    The simulated datasets with $\param{APA} = \SI{85}{\percent}$  are used here.
  }\label{fig:data-mc-size}
\end{figure}


\noindent The Achilles heel of any analysis depending on simulated data is
how well the simulation describes reality.
In this section, observed and simulated image parameter distributions are compared.
Optimizing the simulation configuration is a slow and computationally expensive 
task.
Root causes of disagreements between observed and simulated data are rarely obvious
and recomputing the simulations takes considerable time.

\begin{figure}
  \centering
  \includegraphics[page=4]{build/plots/data_mc_comparison.pdf}
  \caption{%
    Image parameter \param{length} for the three different datasets normalized to
    the expected event rate for one hour of observation time.
  }\label{fig:data-mc-length}
\end{figure}

Before, only protons had been simulated for \gls{fact} and been
reweighted to \eqref{eq:all-particle}, basically making the assumption
that all charged cosmic rays are protons, e.\,g.\ in \cite{phd-temme}.


\begin{figure}
  \centering
  \includegraphics[page=1]{build/plots/data_mc_comparison.pdf}
  \caption{%
    Image parameter \param{concentration\_cog} for the three different datasets normalized to
    the expected event rate for one hour of observation time.
  }\label{fig:data-mc-conccog}
\end{figure}
In \autoref{fig:data-mc-size}, the distributions of the \param{size} parameter,
the total number of reconstructed Cherenkov photons in the cleaned image, is shown
for three datasets:
\begin{enumerate}[nosep]
  \item The observed Crab Nebula observations
  \item Simulated protons using $\param{APA} = \SI{85}{\percent}$ weighted to the all particle spectrum \eqref{eq:all-particle}.
  \item Simulated protons and helium using $\param{APA} = \SI{85}{\percent}$  weighted to their respective individual spectra
    according to \eqref{eq:proton-flux} and \eqref{eq:helium-flux} as shown in \autoref{fig:cr-flux}.
\end{enumerate}
As can be seen, the effect of the composition of cosmic rays is not negligible.
For larger values of \param{size} the agreement between observations and the combined
proton and helium flux is much better than under the only protons assumption.
There are however considerably more events in both simulated distributions at smaller values
of \param{size} compared to the observations.
Mismatches are also observed in the other image parameters, e.\,g.\ \param{length} (\autoref{fig:data-mc-length}) or \param{concentration\_cog} (\autoref{fig:data-mc-conccog}).

\begin{figure}
  \centering
  \includegraphics[page=4]{build/plots/data_mc_comparison_size_gt_500.pdf}
  \caption{%
    Image parameter \param{length} for the three different datasets normalized to
    the expected event rate for one hour of observation time. 
    Only events with $\param{size} ≥ 500$ are shown.
  }\label{fig:data-mc-large-size-length}
\end{figure}

To check whether the mismatches in the other image parameters are correlated to the  
mismatches observed in \param{size},
\autoref{fig:data-mc-large-size-length}, \autoref{fig:data-mc-large-size-conccog} and \autoref{fig:data-mc-large-size-leakage1}
show the image parameters \param{length}, \param{concentration\_cog} and \param{leakage1} only for events with $\param{size} ≥ 500$.
The agreement is much better for these bright events, hinting that the mismatches
mostly affect the lower energy part.
This can have multiple causes, dimmer events have a lower signal to noise ratio,
resulting in not properly simulated noise affecting these events more strongly.
Noise is simulated under the simple assumption of a single white noise component in \ceres{}, 
compare \autoref{sec:ceres}.
An approach to mitigate the simple noise model in \ceres{} is to use measured
noise and superimpose it to simulated shower images without noise to get
a more realistic noise for simulated events. 
A first implementation of this is described in \cite{master-bulinski} and
is currently being evaluated for \gls{fact} simulations in \cite{phd-buss}.


\begin{figure}
  \centering
  \includegraphics[page=1]{build/plots/data_mc_comparison_size_gt_500.pdf}
  \caption{%
    Image parameter \param{concentration\_cog} for the three different datasets normalized to
    the expected event rate for one hour of observation time. 
    Only events with $\param{size} ≥ 500$ are shown.
  }\label{fig:data-mc-large-size-conccog}
\end{figure}

\begin{figure}
  \centering
  \includegraphics[page=3]{build/plots/data_mc_comparison_size_gt_500.pdf}
  \caption{%
    Image parameter \param{leakage1} for the three different datasets normalized to
    the expected event rate for one hour of observation time. 
    Only events with $\param{size} ≥ 500$ are shown.
  }\label{fig:data-mc-large-size-leakage1}
\end{figure}


Figure \autoref{fig:data-mc-size-all} shows the image parameter \param{size} for
all three simulated settings of the additional photon acceptance.
As expected, with more light in the events, more events are triggering and 
overall higher event rates are observed, especially at lower values of \param{size}.

\begin{figure}
  \centering
  \includegraphics[page=6]{build/plots/data_mc_comparison_all.pdf}
  \caption{%
    Image parameter \param{size} for the Crab Nebula observations and
    combined proton and helium flux for all three simulated values of \param{APA}.
  }\label{fig:data-mc-size-all}
\end{figure}


Finally, \autoref{fig:data-gamma-size} shows a comparison between signal candidates
on the observed Crab Nebula data and simulated gammas for the dataset with $\param{APA} = \SI{85}{\percent}$.
This is achieved by applying very loose event selection criteria ($\param{gamma\_prediction} ≥ \num{0.5}$ and $θ^2 ≤ \SI{0.1}{{deg}\squared})$ and subtracting the distribution
for the off regions from the distribution in the on region, leaving only the excess events.
The same distributions for the other settings of \param{APA} are shown in Appendix~\ref{apx:gamma-size}.
Also these figures lead to the conclusion that of the three simulated datasets,
$\param{APA} = \SI{85}{\percent}$ has the best agreement with the observed data,
at least in the higher energy range.

\begin{figure}
  \centering
  \includegraphics{build/plots/data_gamma_comp_apa85.pdf}
  \caption{%
    Image parameter \param{size} for simulated gammas and the observed excess events
    in the on region for loose event selection criteria.
    The simulated datasets with $\param{APA} = \SI{85}{\percent}$  are used here.
  }\label{fig:data-gamma-size}
\end{figure}

These comparisons show, that work is still required, especially for the lower
energy events, to get to full agreement between observed and measured distributions.
Through the addition of helium overall agreement has been greatly improved, allowing
more dedicated searches for the causes of the remaining disagreements in the future.


\section{Detection Significance of the Crab Nebula}\label{sec:detection}
\begin{figure}
  \centering
  \includegraphics{build/plots/skymap_apa85.pdf}
  \caption{%
    Skymap of the reconstructed positions of all events after applying a
    prediction threshold of $t_p = \protect\input{build/threshold_apa85.txt}\unskip$.
    The catalog position of the Crab Nebula is marked using a gray circle.
    The reconstruction models were trained on the $\param{APA} = \SI{85}{\percent}$ dataset.
    A small systematic offset of the center of gravity of the reconstructed events
    vs.\ the catalog position is visible.
  }\label{fig:skymap}
\end{figure}

\noindent After training the four models for the event reconstruction on the respective
simulated datasets, they are now applied to the observed Crab Nebula data
and the simulated datasets for the performance evaluation.
The source dependent parameters like the distance of the reconstructed source
position to the assumed source positions are also computed in this step.
The result are lists of reconstructed events with very few properties
per event, a last dramatic reduction of needed space:
\begin{itemize}[nosep]
  \item Event identifiers
  \item Time of observation
  \item Estimated energy
  \item Estimated source position in equatorial (\gls{icrs}) coordinates.
  \item Score of the gamma-hadron-separation
  \item Pointing direction of the telescope
  \item Angular distances $θ$ to the observed source and the five off positions
\end{itemize}
\gls{fact} uses an event identifier composed of three keys: the night when 
observations were started (the date of the previous evening if after {\lining 00:00}),
the run number within that night and the event number within the run.
Together these three form the unique identifier of an event recored by \gls{fact}.
Simulations only use the corsika run number and the event number.

As described in \autoref{sec:wobble}, \gls{fact} observes sources in wobble mode.
This allows simultaneous estimation of the background rate at positions
in the field of view geometrically identical to the source position.
All events within an angular distance of $θ_{\max}$ of the suspected source position
are counted as \enquote{on} events and all events within that radius of any of
the off positions are counted as \enquote{off} events, used for estimation of the remaining background.

The statistical significance of a source detection can be calculated using
the likelihood ratio test first proposed by Li and Ma in \cite{lima} testing
the null hypothesis of no source of gamma rays.
The null hypothesis is rejected with a significance in units of standard deviations $\sigma$
given by
\begin{equation}
  S = \sqrt{2} \left(
    \Non \ln\!\left(\frac{1 + α}{α} \cdot \frac{\Non}{\Non + \Noff}\right)
    + \Noff \ln\!\left((1 + α) \cdot \frac{\Noff}{\Non + \Noff}\right)
  \right)^{\!\sfrac{1}{2}}\label{eq:lima} \text{\cite[(17)]{lima}}
\end{equation}
where \Non{} is the number of events recorded in the on region,
\Noff{} is the number of events in all off regions and $α$ is the size of
the on region divided by the size of the Off region.
The \facttools{} analysis uses 5 identical Off regions, so $α = \num{0.2}$.
By convention in particle physics, \SI{3}{\sigma} are interpreted as a strong hint
and \SI{5}{\sigma} are called a detection, corresponding to false alarm probabilities
of \num{1.3e-3} and \num{3e-7}, respectively.
In the case for this analysis however, the goal is not detecting a new, unknown
source but to estimate the potential of the telescope and the analysis.

\begin{figure}
  \centering
  \includegraphics{build/plots/significances_apa85.pdf}
  \caption{%
    Significance for the detection of the Crab Nebula for a grid-search
    of event selection parameters using the simulated data with $\param{APA} = \SI{85}{\percent}$
    for training the models. 
    $t_p$ was varied between \num{0.5} and \num{1.0} in steps of \num{0.01} and
    $\theta^2_{\max}$ was varied between \num{0} and \num{0.1} in steps of \num{0.02}.
    The best event selection in terms of detection significance according to \eqref{eq:lima}
    is marked using the black circle.
  }\label{fig:significances}
\end{figure}

The final step is applying the prediction threshold $t_p$ for the gamma-hadron-separation
and choosing the radius of the on and off regions $\theta_{\max}$.
This is done using a grid-search, shown in \autoref{fig:significances}, calculating the
significance for each combination of $t_p$ and $\theta_{max}$.

\begin{figure}
  \centering
  \includegraphics{build/plots/theta2_apa85.pdf}
  \caption{%
    Theta-Squared-Plot for the Crab Nebula dataset reconstructed using the models
    trained on the simulated dataset using $\param{APA} = \SI{85}{\percent}$ and
    using the $t_p$ and $\theta_{\max}$ yielding the highest significance from \autoref{fig:significances}. 
  }\label{fig:theta2}
\end{figure}

With a maximum significance of over \SI{60}{\sigma}, this is a major improvement over the 
previous analyses. 
On the same dataset, the RapidMiner based analysis achieved \SI{39.9}{\sigma}~\cite[72]{phd-temme}
and an analysis using the \texttt{aict-tools}—before the
new \param{disp} estimation was implemented—achieved \SI{43.3}{\sigma}~\cite{icrc-performance}.
The onsite quick-look-analysis based on \gls{mars} achieved \SI{41.3}{\sigma}\footnote{Results taken from \gls{fact}'s database.}.

In gamma-ray astronomy, the most iconic visualisation for the detection of a source
is the $θ²$-Plot. 
It shows the distribution of the angular distances to the assumed source position
for both the source of interest and for all off positions combined.
In case a detection is made, an excess of events in the on region above the off
region is expected for very small values of $θ²$.
This can be observed beautifully in \autoref{fig:theta2}.

\begin{table}[htpb]
  \tablefont
  \centering
  \caption{
    Event numbers after each step of the analysis for the five different datasets.
    Numbers after applying the threshold and the on region selection are those for
    the proton and gamma simulations are those of the test datasets, i.\,e.\ only for \SI{75}{\percent} 
    of the events.
    No test dataset is created for the diffuse gamma rays.
  }
  \label{tab:events}
  \makebox[\textwidth]{%
    \input{build/event_num_table.tex}
  }
\end{table}

In \autoref{tab:events}, the number of events after each analysis step is shown
for all datasets.

\section{Separation Performance}

The random forest classifiers to separate gamma ray induced showers from
hadron induced background were trained using the \texttt{aict-tools} configuration
in Appendix~\ref{apx:aict-config}.
It uses \num{500000} samples each of the signal and background class and 18 features
in total.
A useful property of tree based models is their ability to provide information 
about which features contributed the most to the decision making.
\autoref{fig:feature-importance-sep} shows the feature importances for each
of the 200 decision trees in each of the 10 cross validation steps.

\begin{figure}
  \centering
  \includegraphics[page=4]{build/plots/separator_performance_apa85.pdf}
  \caption{%
    Feature importances for the classification model.
    Each point represents one tree in one of the cross validation iterations,
    the box plots show the median as blue line, the size of the box goes 
    from the first to the third quartile and the whiskers extend.
    The feature \param{area\_size\_cut\_var} has been used for manual 
    event selection in the \gls{mars} analysis and is defined as $\param{area} / (\log{\param{size}})^2$, where $\param{area}=\PI \cdot \param{width} \cdot \param{length}$.
    \param{size\_area} is defined as \param{size} divided by \param{area}.
    Both of these features are generated automatically by the \texttt{aict-tools} from
    the input features described in \autoref{sec:parameters}.
  }\label{fig:feature-importance-sep}
\end{figure}

The \gls{roc}-curves for each of the cross validation iterations and the mean
are shown in \autoref{fig:classifier-roc}.
\autoref{fig:roc-vs-energy} shows the area under the \gls{roc}-curve in intervals
of estimated gamma energy as predicted by the energy prediction model.
It can be seen that the larger the estimated energy, the better
the model can distinguish between gammas and protons in the same energy interval.

\begin{figure}
  \begin{captionbeside}{%
    \gls{roc} curves of the classification model.
    Because the number of training samples is very large, variance between
    cross validation iterations is low and the single curves are only barely visible.
    \label{fig:classifier-roc}%
  }
  \raisebox{\dimexpr\baselineskip-\totalheight\relax}{%
    \includegraphics[page=1, trim={0 0 5.25cm 0}]{build/plots/separator_performance_apa85.pdf}
  }
  \end{captionbeside}
\end{figure}


\begin{figure}
  \centering
  \includegraphics{build/plots/roc_vs_energy_apa85.pdf}
  \caption{%
    Area under the \gls{roc}-curve evaluated in bins of estimated gamma energy $E_\text{est}$.
    The higher the estimated energy, the better the classifier performance.
    For a discussion, why bins of estimated energy are used see \autoref{sec:sensitivity}.
  }\label{fig:roc-vs-energy}
\end{figure}

In the next section, the effective collection area for gammas and protons is shown 
in \autoref{fig:effective-area-bg}, demonstrating that applying the 
selection criteria dramatically decreases collection area for protons while keeping
most of the gamma rays.


\section{Instrument Response Functions}\label{sec:irf}

\Glspl{irf} are key properties of the combined system
of detector and analysis software that describe the response to particles given
their physical attributes.
The \glspl{irf} are needed to do any physics analysis from lists
of events with reconstructed properties like estimated energy, particle type and origin
with additional metadata such  as measurement time and observed source.
Four different \gls{irf} components together fully describe the measurement process of any \gls{iact}:
\begin{itemize}
  \item Effective area, \autoref{sec:effective-area}, describes the detection
    efficiency of a telescope as the area a perfect detector directly measuring 
    gamma rays with the same efficiency as the real telescope would have.  
  \item Energy dispersion, \autoref{sec:energy-dispersion}, describes
    the performance of the energy reconstruction as the probability density 
    to get an estimated energy $\Eest$ for a given true energy $\Etrue$.
  \item \Gls{psf} of the gamma ray origin reconstruction, \autoref{sec:psf},
    not to be confused with the optical point spread function of the reflector system,
    describes how well the position of a gamma ray can be reconstructed.
  \item Background rate describes the number of background events still present after
    applying the event selection after the gamma-hadron-separation.
\end{itemize}

For a long time, exact definition and treatment of these were slightly different 
from experiment to experiment but with the advent of \gls{cta} and it being
operated as an open observatory, a standardization effort has started
 under the name \enquote{Data Formats for Gamma-Ray Astronomy}~\cite{ogadf}.
In their most general form, called full-enclosure-\glspl{irf} by \cite{ogadf},
all these depend on the energy, either true or estimated, and the offset of the reconstructed
source position from the optical axis, 
relying on the assumption that the \glspl{irf} are radially symmetric.

In case of observing point sources in wobble mode, the \glspl{irf} are simpler,
as only one offset from the optical axis needs to be given,
the background can be estimated from the same observations and the effective area
includes the selection of the on region. 
The \gls{psf} only plays a role in the selection of the on region's size and
is not needed afterwards.
Only these point-like \glspl{irf} will be discussed here, as they are
the most relevant for \gls{fact}'s monitoring of well-known, bright point sources.

The calculation of these \glspl{irf} was implemented in a python package of the 
same name, again by Kai Brügge and myself\footnote{\url{https://github.com/fact-project/irf}}.
The uncertainties are estimated by applying the same calculations to 
datasets created by sampling with replacement from the original dataset, bootstrapping.

\subsection{Angular resolution}\label{sec:psf}
\begin{figure}
  \centering
  \includegraphics{build/plots/disp_metrics_apa85.pdf}
  \caption{%
    Accuracy of the prediction for $\sgn{\param{disp}}$ and $r^2$ score 
    of the prediction for $\abs{\param{disp}}$ calculated in bins of true energy 
    for the point source gammas observed in wobble mode before and after applying
    the prediction threshold.
  }\label{fig:disp-metrics}
\end{figure}

\noindent The angular resolution is a measure for how well an analysis can reconstruct
the origin of a gamma ray.
As the \param{disp} parameterization is now done using the \texttt{aict-tools},
the used features and the parameters for the random forest are stored
in the configuration file in Appendix~\ref{apx:aict-config}.
\autoref{fig:disp-metrics} shows the performance metrics of the \param{disp} estimation
in several energy ranges for simulated gamma-ray events before and after applying the
prediction threshold.
Both accuracy and $r^2$ score improve with higher energies
and gamma-ray events that earn a high $\param{gamma\_prediction}$ are also easier to
reconstruct for the \param{disp} prediction.


Compared to the older \facttools{} parameterization, the angular resolution
has drastically improved, especially in the higher energies.
While at the lower energies more events are lost due to misclassification of
$\sgn\param{disp}$, the background rate was reduced by a factor of two as 
the diffuse background is not skewed towards the assumed source position anymore.

Angular resolution is defined here as the radius from the source position
which contains \SI{68}{\percent} of the reconstructed events.

\begin{figure}
  \centering
  \includegraphics{build/plots/compare_old_disp_only_correct.pdf}
  \caption{%
    Comparison of the angular resolution using the simple \param{disp} parameterization
    from \facttools{} with the new, machine learning based parameterization implemented
    in the \texttt{aict-tools}.
    Large improvements  could be achieved over the full energy range,
    but especially at the higher energies.
    Only events with the correct prediction for $\sgn{\param{disp}}$ and
    with $\param{gamma\_prediction} \geq \protect\input{build/theta2_cut_apa85.txt}\unskip$ were used. 
    A version where all events were used is shown in Appendix~\ref{apx:old-disp-all},
    showing that for small energies,
    a lot of events are essentially lost because of the wrong sign prediction.
    This can also be seen in \autoref{fig:disp-metrics} showing the accuracy of the sign prediction and in \autoref{fig:effective-area} when comparing
    the effective area before and after applying the event selection based on $\theta$.
  }\label{fig:ang-comp}
\end{figure}


\autoref{fig:ang-comp} shows the angular resolution of the dataset with $\param{APA}= \SI{85}{\percent}$ for the old disp parameterization \eqref{eq:disp} compared 
to the new random forest based approach.
The random forest based approach improves the angular resolution over the full
energy range, but the improvement is particularly large for energies higher than
\SI{5}{\TeV}.


\subsection{Effective Area}\label{sec:effective-area}

The differential (in energy) flux of a gamma-ray point source has the dimensionality particles per
time, area and energy.
As a telescope counts gamma rays, the observed area, the observation time and 
the energy of the particles needs to be known to estimate flux.
As a real telescope is also unable to detect all gamma rays that reach the telescope,
it is necessary to know the detection efficiency or acceptance to correctly
reconstruct the original number of gamma rays.
In gamma ray astronomy, it is common to define an effective area $\Aeff$ that
combines the area the detector is observing with the detection efficiency.
It can be interpreted as the area a perfect detector directly observing the gamma
rays would need to observe the same number of particles.
Effective area can only be expressed as function of the true, simulated energy,
as it relates the number of detected particles to the number of all particles and
no estimated energy exists for the non-detected particles.

\begin{figure}
  \centering
  \includegraphics{build/plots/effective_area_apa85.pdf}
  \caption{%
    Effective area in bins of simulated energy for gamma rays from a point source observed in wobble mode
    after the event pre-selection, after applying the prediction threshold and
    after applying both the prediction threshold and the selection using $θ^2_{\max}$.
    The efficiency of the classifier results in an overall drop in effective area 
    while the $θ²$ cut mainly reduces effective area for low energies. 
  }\label{fig:effective-area}
\end{figure}

Effective area is then defined as
\begin{align}
  \Aeff(E) &=  p(E) A \\
  \intertext{%
    where $p(E)$ is the detection probability for a gamma ray with energy $E$ and $A$
    is the area the telescope is able to observe.
    $p(E)$ is estimated by counting detected events $N_\text{detected}$ and
    divide by the total number of simulated events $N_\text{simulated}$ in intervals of $E$.
    Thus the discretized effective area in the $i$-th energy interval is: 
  }
  \Aeff_{, i} &=  \frac{N_{\text{detected}, i}}{N_{\text{simulated}, i}} A_\text{simulated}
\end{align}
$A_\text{simulated} = \PI \Rmax[2]$ is the area the impact points of the gamma rays were simulated in.
The simulation has to sample the whole area, the telescope is sensitive to,
as otherwise the effective area will be underestimated.
\autoref{fig:impact} shows the distribution of impact distance of the simulated gamma rays after
applying the image cleaning.

\begin{figure}
  \centering
  \includegraphics{build/plots/impact.pdf}
  \caption{%
    Distribution of the impact parameter, the distance from where the shower axis
    intersects the observation level to the telescope. 
    Because \gls{corsika} draws the impact scattering perpendicular to the shower axis
    but measures the impact point on the ground level, the cutoff is not hard
    at $R_{\max} = \SI{300}{\meter}$ but smeared out a little due to events with zenith
    distances ${} > 0$. 
    The number of detected events rises linearly first,
    with the increasing area enclosed by the respective ring,
    but drops of after roughly \SI{130}{\meter},
    which is related closely to the radius in which the main part of the Cherenkov light reaches
    the observation level, compare \autoref{fig:footprints}.
  }\label{fig:impact}
\end{figure}


The effective area can be evaluated for different steps in the analysis, where
each step is additionally discarding events:
\begin{enumerate}
  \item Events that triggered the telescope, see \autoref{sec:camera}.
  \item Events with more than five pixels after image cleaning, see \autoref{sec:cleaning}.
  \item Events surviving the event pre-selection, see \autoref{sec:precuts}.
  \item Events after applying the gamma-hadron-separation threshold $p_\gamma > t_p$, see \autoref{sec:classification}.
  \item Events in the On region after applying $\theta^2 < \theta^2_{\max}$, see \autoref{sec:detection}.
\end{enumerate}

Effective area for the last three steps is shown in \autoref{fig:effective-area} 
for gamma rays from a point source observed in wobble mode. 
\autoref{fig:effective-area-bg} shows the effective area for step 3 and 5 
also for the background class.
The shape is as expected in gamma-ray astronomy, first rising steeply with
energy and than reaching a plateau with a slight drop off at the highest energies
caused by the growing percentage of showers not fully contained in the camera.

It is only possible to evaluate effective area with respect to the
true gamma-ray energy as no estimated energy is known for events
that did not trigger the telescope, survive the image cleaning or were discarded
by the event pre-selection.


\begin{figure}
  \centering
  \includegraphics{build/plots/effective_area_apa85_bg.pdf}
  \caption{%
    Effective area in bins of simulated energy for gamma rays from a point source observed in wobble mode and protons after the event pre-selection (transparent)
    and after applying both the prediction threshold and the selection using $θ^2_{\max}$ (solid).
    The first energy bin did not contain any proton events after applying the
    event selection criteria and is thus not shown in the logarithmic scale.
    While overall roughly \SI{25}{\percent} of the gamma rays are kept, 
    the proton background is reduced by at least three orders of magnitude.
  }\label{fig:effective-area-bg}
\end{figure}


\subsection{Energy Dispersion}\label{sec:energy-dispersion}

Energy dispersion or migration is the probability density function 
of measuring an estimated energy $\Eest$ given a true energy $\Etrue$.
Knowledge of the energy dispersion is necessary as the energy estimation 
is not perfect and in general has non-zero bias and resolution.
The energy dispersion is thus needed for the statistical reconstruction of spectra
from lists of reconstructed events. 
Methods for this will be discussed in \autoref{chp:spectrum}.
As the energy dispersion closely relates to the other performance metrics of the energy estimation,
these will also be discussed here.

The logarithm of the energy is estimated using a random forest regressor with the configuration
in Appendix~\ref{apx:aict-config}.
It is trained on \num{500000} diffuse gamma rays and comprises \num{200} single
decision trees.
The depth of the trees is limited and a minimum leaf size is set to prevent overtraining
and overly large models, also achieving faster application times.

\begin{figure}
  \centering
  \includegraphics{build/plots/energy_migration_apa85.pdf}
  \caption{%
    Energy migration matrix for gamma rays from a point source observed in
    wobble mode after applying the event selection.
  }\label{fig:energy-migration}
\end{figure}

On the cross validated training dataset, the regressor achieved an $r^2$-score of
\input{build/r2_apa85.tex}\unskip.
The migration matrix is shown in \autoref{fig:energy-migration}.
As the $r^2$-score is a single number over the full energy range used for training and
dependent on the simulated spectrum in case the performance is not uniform over the 
energy range, two metrics are calculated
in bins of true energy, both aggregating the relative error 
\begin{equation}
  \increment E_{\text{rel}} = \frac{\Eest - \Etrue}{\Etrue}.
\end{equation}
\emph{Bias} is here defined as the median of the relative error and \emph{resolution}
as half the interquantile distance between the upper and lower quantile containing
\SI{68.2}{\percent}, as the \SI{1}{\sigma} interval of the normal distribution, of the events:
\begin{equation}
  \operatorname{Resolution} = \frac{Q_{84.1}(\increment E_{\text{rel}}) - Q_{15.9}(\increment E_{\text{rel}})}{2}
\end{equation}

Several other definitions\footnote{See \url{https://xkcd.com/927}} exist and are used in publications,
\cite{magic-performance} is fitting a normal distribution to the histogram of the relative error.
This however underestimates the resolution in case of skewed distributions,
which occur regularly.
\begin{figure}
  \centering
  \includegraphics{build/plots/bias_resolution_apa85.pdf}
  \caption{%
    Bias and resolution for gamma rays from a point source observed in wobble mode
    after applying the event selection.
  }\label{fig:bias-resolution}
\end{figure}
\gls{cta} uses the interval centered around {\lining 0} containing \SI{68.2}{\percent}
of the events, which intertwines centrality and width of the relative error
and can be interpreted as a quantile based version of the root mean squared.
Just taking the mean and standard deviation is prone to be influenced heavily by
few very large outliers, so the median and interquantile distance is taken as a 
more robust method of estimating bias and resolution.
As with the angular resolution, the performance of the energy estimation improves
quite dramatically after discarding gamma rays with low values for $p_\gamma$ and
only taking events in the On region into account.
Bias and resolution for the events selected into the final dataset are shown in 
\autoref{fig:bias-resolution}, Appendix~\ref{apx:bias-resolution-all} shows bias and
resolution without applying the event selection.

\section{Differential Flux Sensitivity}\label{sec:sensitivity}
The central measure of the performance of an \gls{iact} is its
differential flux sensitivity, which is defined as the smallest flux the
telescope can still detect with a certain statistical significance in a given
observation time.
Most commonly, \gls{iact} sensitivities are given for \SI{5}{\sigma} according
to \eqref{eq:lima} in \SI{50}{\hour} of observation time.
As calculating the sensitivity requires counting events, the differential sensitivity 
is given as integral sensitivity in bins of energy.

\begin{figure}
  \centering
  \includegraphics{build/plots/size_vs_true_energy.pdf}
  \caption{%
    \param{size} vs.\ primary energy for simulated proton and gamma ray induced showers.
    The contours contain the listed percentage of all events.
    It can be seen, that protons need to have an approximately five times higher energy
    to produce the same amount of Cherenkov light in the camera of an \gls{iact}.
  }\label{fig:size-energy-comp}
\end{figure}

A controversial topic is whether the sensitivity should be calculated in bins
of $\Etrue$ or bins of $\Eest$.
\autoref{fig:size-energy-comp} shows the amount of Cherenkov light the \gls{fact} 
camera received, image parameter \param{size}, against the energy of the primary particle
for gamma rays and protons.
It is clear, that protons of the same primary energy as gamma rays do not produce
similar images.
The same can be seen in Appendix~\ref{apx:est-vs-true} which shows the same for the
estimated gamma energy vs true energy.
Consequently, calculating sensitivity in bins of true energy for both 
gammas and protons uses a very unrealistic background model and is also not
possible to calculate on observed data.
On the other hand, calculating sensitivity in bins of estimated energy is
possible on both simulated and observed data and provides a more realistic background
modeling in each bin.
The disadvantage is  that the behaviour of the energy migration plays a role in 
the discretization and that energy migration has to be taken into account to answer
the question  if a source with a certain flux would be detectable.
A solution to both problems could be either to unfold the event counts using the energy migration,
assuming unfolding the hadronic background yields a background estimation in bins
of true energy or to only discretize the gamma rays in bins of true energy and
find a better model for the background in these bins.

For now, we look into the simulated and observed sensitivity in bins
of estimated energies.
In each bin, the number of events in the on region $\Non$ and the number
of events in the off regions $\Noff$ is calculated.
The number of signal events expected from a source with flux of a fraction $p$ of the
Crab Nebula flux is estimated to be
\begin{equation}
  N_{\text{signal}} = \frac{t_{\text{ref}}}{{t_\text{obs}}} (p \cdot (\Non - α \Noff)),
\end{equation}
where $t_\text{ref}$ is the reference observation time for the sensitivity,
$t_\text{obs}$ is the observation time of the dataset and $α = 0.2$ is the
size ratio of the on region to the Off regions.
The background expectancy only needs to be scaled to the correct reference time:
\begin{equation}
  N_{\text{background}} = \frac{t_{\text{ref}}}{{t_\text{obs}}} \Noff
\end{equation}
The scaled event counts in the off and on region are then
\begin{align}
  N'_{\text{on}} &= \frac{t_{\text{ref}}}{{t_\text{obs}}} (p \cdot (\Non - α \Noff) + α \Noff) \\
  N'_{\text{off}} &= \frac{t_{\text{ref}}}{{t_\text{obs}}} \Noff
\end{align}
The flux sensitivity is then the value of $p$ that solves
\begin{equation}
  S_{\text{Li \& Ma}} (N'_{\text{on}}, N'_{\text{off}}, α) \stackrel{!}{=} 5. \label{eq:sensitivity}
\end{equation}
Unfortunately, finding $p$ analytically is impossible and is thus done numerically
using the Newton--Raphson method, the results are shown in \autoref{fig:sensitivity}.

\begin{figure}
  \centering
  \includegraphics{build/plots/sensitivity_apa85.pdf}
  \caption{%
    Differential sensitivity in bins of estimated energy $E_\text{est}$ for
    simulated and observed data. 
    For the simulations, the gamma-ray events were weighted to the 
    flux of the Crab Nebula as measured and parameterized using a log-parabola 
    by the \gls{magic} telescopes in \cite{magic-crab}. 
    The simulated protons were weighted to the all particle cosmic rays spectrum
    \eqref{eq:all-particle}.
    While the simulated spectral sensitivities agree within statistical uncertainties
    for energies larger than \SI{2}{\TeV},
    a disagreement is visible for the lower energies.
    The lowest energy bin is missing for the simulated dataset,
    as too few events remained here for estimating the sensitivity.
  }\label{fig:sensitivity}
\end{figure}

For the simulated datasets, the gamma rays are again weighted to \eqref{eq:magic-crab}.
Due to the limited helium core statistics, see \autoref{tab:events}, the background
is only modeled by protons weighted to \eqref{eq:all-particle}.
The sensitivity is calculated in five bins equidistant in log-space per decade in estimated energy
for \SI{50}{\hour} of observation time.

The integral sensitivity over the whole energy range is \input{build/integral_sensitivity_apa85.tex}\unskip{} of the Crab Nebula flux for the observed data over the whole energy range,
which is a considerable improvement over the \SI{15.2}{\percent}\footnote{Calculated
using \eqref{eq:sensitivity} from the \Non{} and \Noff{} given in Table 6.7, p.~72} in \cite{phd-temme}
and \SI{13.7}{\percent} from \cite{icrc-performance}.
