\subsection{Comparison of Observations and Simulations}

\begin{figure}
  \centering
  \includegraphics[page=6]{build/plots/data_mc_comparison.pdf}
  \caption{%
    Image parameter \param{size} for the three different datasets normalized to
    the expected event rate for one hour of observation time.
    The simulated datasets with $\param{APA} = \SI{85}{\percent}$  are used here.
  }\label{fig:data-mc-size}
\end{figure}


\noindent The Achilles heel of any analysis depending on simulated data is
how well the simulation describes reality.
In this section, observed and simulated image parameter distributions are compared.
Optimizing the simulation configuration is a slow and computationally expensive 
task.
Root causes of disagreements between observed and simulated data are rarely obvious
and recomputing the simulations takes considerable time.

\begin{figure}
  \centering
  \includegraphics[page=4]{build/plots/data_mc_comparison.pdf}
  \caption{%
    Image parameter \param{length} for the three different datasets normalized to
    the expected event rate for one hour of observation time.
  }\label{fig:data-mc-length}
\end{figure}

Before, only protons had been simulated for \gls{fact} and been
reweighted to \eqref{eq:all-particle}, basically making the assumption
that all charged cosmic rays are protons, e.\,g.\ in \cite{phd-temme}.


\begin{figure}
  \centering
  \includegraphics[page=1]{build/plots/data_mc_comparison.pdf}
  \caption{%
    Image parameter \param{concentration\_cog} for the three different datasets normalized to
    the expected event rate for one hour of observation time.
  }\label{fig:data-mc-conccog}
\end{figure}
In \autoref{fig:data-mc-size}, the distributions of the \param{size} parameter,
the total number of reconstructed Cherenkov photons in the cleaned image, is shown
for three datasets:
\begin{enumerate}[nosep]
  \item The observed Crab Nebula observations
  \item Simulated protons using $\param{APA} = \SI{85}{\percent}$ weighted to the all particle spectrum \eqref{eq:all-particle}.
  \item Simulated protons and helium using $\param{APA} = \SI{85}{\percent}$  weighted to their respective individual spectra
    according to \eqref{eq:proton-flux} and \eqref{eq:helium-flux} as shown in \autoref{fig:cr-flux}.
\end{enumerate}
As can be seen, the effect of the composition of cosmic rays is not negligible.
For larger values of \param{size} the agreement between observations and the combined
proton and helium flux is much better than under the only protons assumption.
There are however considerably more events in both simulated distributions at smaller values
of \param{size} compared to the observations.
Mismatches are also observed in the other image parameters, e.\,g.\ \param{length} (\autoref{fig:data-mc-length}) or \param{concentration\_cog} (\autoref{fig:data-mc-conccog}).

\begin{figure}
  \centering
  \includegraphics[page=4]{build/plots/data_mc_comparison_size_gt_500.pdf}
  \caption{%
    Image parameter \param{length} for the three different datasets normalized to
    the expected event rate for one hour of observation time. 
    Only events with $\param{size} ≥ 500$ are shown.
  }\label{fig:data-mc-large-size-length}
\end{figure}

To check whether the mismatches in the other image parameters are correlated to the  
mismatches observed in \param{size},
\autoref{fig:data-mc-large-size-length}, \autoref{fig:data-mc-large-size-conccog} and \autoref{fig:data-mc-large-size-leakage1}
show the image parameters \param{length}, \param{concentration\_cog} and \param{leakage1} only for events with $\param{size} ≥ 500$.
The agreement is much better for these bright events, hinting that the mismatches
mostly affect the lower energy part.
This can have multiple causes, dimmer events have a lower signal to noise ratio,
resulting in not properly simulated noise affecting these events more strongly.
Noise is simulated under the simple assumption of a single white noise component in \ceres{}, 
compare \autoref{sec:ceres}.
An approach to mitigate the simple noise model in \ceres{} is to use measured
noise and superimpose it to simulated shower images without noise to get
a more realistic noise for simulated events. 
A first implementation of this is described in \cite{master-bulinski} and
is currently being evaluated for \gls{fact} simulations in \cite{phd-buss}.


\begin{figure}
  \centering
  \includegraphics[page=1]{build/plots/data_mc_comparison_size_gt_500.pdf}
  \caption{%
    Image parameter \param{concentration\_cog} for the three different datasets normalized to
    the expected event rate for one hour of observation time. 
    Only events with $\param{size} ≥ 500$ are shown.
  }\label{fig:data-mc-large-size-conccog}
\end{figure}

\begin{figure}
  \centering
  \includegraphics[page=3]{build/plots/data_mc_comparison_size_gt_500.pdf}
  \caption{%
    Image parameter \param{leakage1} for the three different datasets normalized to
    the expected event rate for one hour of observation time. 
    Only events with $\param{size} ≥ 500$ are shown.
  }\label{fig:data-mc-large-size-leakage1}
\end{figure}


Figure \autoref{fig:data-mc-size-all} shows the image parameter \param{size} for
all three simulated settings of the additional photon acceptance.
As expected, with more light in the events, more events are triggering and 
overall higher event rates are observed, especially at lower values of \param{size}.

\begin{figure}
  \centering
  \includegraphics[page=6]{build/plots/data_mc_comparison_all.pdf}
  \caption{%
    Image parameter \param{size} for the Crab Nebula observations and
    combined proton and helium flux for all three simulated values of \param{APA}.
  }\label{fig:data-mc-size-all}
\end{figure}


Finally, \autoref{fig:data-gamma-size} shows a comparison between signal candidates
on the observed Crab Nebula data and simulated gammas for the dataset with $\param{APA} = \SI{85}{\percent}$.
This is achieved by applying very loose event selection criteria ($\param{gamma\_prediction} ≥ \num{0.5}$ and $θ^2 ≤ \SI{0.1}{{deg}\squared})$ and subtracting the distribution
for the off regions from the distribution in the on region, leaving only the excess events.
The same distributions for the other settings of \param{APA} are shown in Appendix~\ref{apx:gamma-size}.
Also these figures lead to the conclusion that of the three simulated datasets,
$\param{APA} = \SI{85}{\percent}$ has the best agreement with the observed data,
at least in the higher energy range.

\begin{figure}
  \centering
  \includegraphics{build/plots/data_gamma_comp_apa85.pdf}
  \caption{%
    Image parameter \param{size} for simulated gammas and the observed excess events
    in the on region for loose event selection criteria.
    The simulated datasets with $\param{APA} = \SI{85}{\percent}$  are used here.
  }\label{fig:data-gamma-size}
\end{figure}

These comparisons show, that work is still required, especially for the lower
energy events, to get to full agreement between observed and measured distributions.
Through the addition of helium overall agreement has been greatly improved, allowing
more dedicated searches for the causes of the remaining disagreements in the future.
