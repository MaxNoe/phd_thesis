\appendix
\addtocontents{toc}{\protect\setcounter{tocdepth}{0}}

\chapter{Appendix}
\raggedbottom
\captionsetup{hypcap=false}

\section{On Reproducing this Work}\label{apx:reproducing}

This doucment is build completely without human intervention from only a few input 
data files, scripts to create visualizations and the \LaTeX{} source code.
The input data files comprise the image parameters for the different datasets
as described in the thesis and some additional files, e.\,g.\ the \gls{corsika}
output for the Cherenkov footprint visualizations.

To build this thesis, including training all the machine learning models,
applying them to the data, calculate and visualize the performance metrics,
unfold the spectrum of the Crab Nebula and compile the \LaTeX{} document, 
only a single call to the \texttt{make} program is necessary.

\subsection{Source Code of the Thesis}

The source code of this thesis is publicly available at \url{https://github.com/maxnoe/phd_thesis}.
Download an archive from the website or use git:
\begin{lstlisting}
$ git clone https://github.com/maxnoe/phd_thesis 
\end{lstlisting}

\subsection{Access to the Data}

The input data files needed to build this thesis are at the time 
of writing archived on the storage server \texttt{big-tank.app.tu-dortmund.de}
in the directory
\begin{center}
\texttt{/POOL/users/mnoethe/phd\_thesis/data}.
\end{center}
This directory is required to be copied or linked into the \texttt{data} directory
in the base directory of this thesis.


\subsection{Used Software}

\texttt{GNU Make}~\cite{make} is used to define the steps needed
to produce this thesis. 
The analysis itself builds on the
scientific Python stack~\cite{numpy,scipy,pandas,matplotlib,scikit-learn},
astropy~\cite{astropy} and a number of packages developed and contributed
to by me during the work on this thesis.

Building the thesis needs TeXLive 2019, from \url{https://www.tug.org/texlive},
for installation instructions see \url{https://toolbox.pep-dortmund.org/install/linux}.

Python in version 3.7 via the conda package manager is used.
On Linux, to download and install miniconda and setup the environment needed
for this thesis, use:

\begin{lstlisting}
$ curl -LO https://repo.anaconda.com/miniconda/Miniconda3-4.7.12-Linux-x86_64.sh
$ bash Miniconda3-4.7.12-Linux-x86_64.sh -b -p path/to/install/miniconda
$ source path/to/install/miniconda/etc/profile.d/conda.sh
$ conda env create -n phd_mnoethe -f full_environment.yaml
\end{lstlisting}
The \texttt{full\_environment.yaml} file lists all packages in the exact
version as was used to produce this thesis.
The file \texttt{environment.yaml} lists the minimal, direct dependencies 
with more lax version requirements, these are:
\lstinputlisting{../environment.yaml}

\subsection{Building this thesis}

After installing all required software and acquiring the input data,
the only thing needed to build this document and all its figures and tables
is to call
\begin{lstlisting}
$ conda activate phd_mnoethe
$ make
\end{lstlisting}
in the base directory of the git repository.
It will take a few hours, to train all the machine learning models,
apply them, create figures and tables and to typeset the final document.

\subsection{Docker Container}

A docker image with all installed software to directly build this thesis
is defined in the \texttt{Dockerfile} in the git repository and is stored
on the \texttt{big-tank} server. 

\section{Cherenkov Footprint of 10 TeV Iron Shower by Emitting Particle}
\begin{center}
  \includegraphics{build/plots/iron_footprint_rgb.pdf}
  \captionof{figure}{%
    Cherenkov light at the observation level for a \SI{10}{\TeV} iron nucleus.
    The color channels are for the different emitting particles:
    red is light caused by electrons and positrons, green for muons and blue
    for all others, mainly protons and pions.
  }\label{apx:iron-rgb}
\end{center}

\section{Covariance Matrices of the Cosmic Ray Flux Fits}\label{apx:fit-results}

For the proton flux:
\begin{equation}
  \operatorname{Cov}(γ, \log_{10}(Φ_0)) = \input{build/proton_cov.tex}
\end{equation}
For the helium flux:
\begin{equation}
  \operatorname{Cov}(γ, \log_{10}(Φ_0)) = \input{build/helium_cov.tex}
\end{equation}

\section{Probability Density Functions for Primary Properties in CORSIKA}\label{apx:corsika-pdfs}
The energy is sampled from a power law distribution with the following \gls{pdf}:
\begin{equation}
  P_E(E) = \begin{cases}
    \frac{(γ + 1) \Eref[γ]}{\Emax[γ + 1] - \Emin[γ + 1]} \left(\frac{E}{\Eref}\right)^γ &
    \Emin ≤ E ≤ \Emax \\
    0 &
    \text{otherwise}
  \end{cases}
\end{equation}
With $\Eref = \SI{1}{\GeV}$ for \gls{corsika}.
The impact distance is sampled such that it is uniform in area, resulting in the following \gls{pdf}:
\begin{equation}
  P_R(R) =  \begin{cases}
    \frac{2R}{R_{\max}^2} & 0 ≤ R ≤ R_{\max} \\
    0 & \text{otherwise}
  \end{cases}
\end{equation}
The azimuth angle is sampled uniformly:
\begin{equation}
  P_φ(φ) = \begin{cases}
    \frac{1}{φ_{\max} - φ_{\min}} & φ_{\min} ≤ φ ≤ φ_{\max} \\
    0 & \text{otherwise}
  \end{cases}
\end{equation}
The zenith distance is sampled such that events are uniform in solid angle in the
given limits:
\begin{equation}
  P_ϑ(ϑ) = \begin{cases}
    \frac{\sinϑ}{\cosϑ_{\min} - \cosϑ_{\max}} & ϑ_{\min} ≤ ϑ ≤ ϑ_{\max} \\
    0 & \text{otherwise}
  \end{cases}
\end{equation}
This results in a sawtooth-like distribution, when several zenith intervals
are simulated.


\section{Database Tables of Mopro3}\label{apx:mopro-tables}

\subsection{CorsikaSettings}

\begin{description}[font=\ttfamily, itemsep=0.2ex, parsep=0ex]
  \item[id, Integer, PrimaryKey]
  \item[name, Text] Human readable description, used for directory structure and output files.
  \item[version, Integer] \gls{corsika} version identifier. The current release is 77000,
    used for production was 76900.
  \item[config\_h, Text] The \gls{corsika} compilation time configuration as precompiler header,
    this is generated by running the \gls{corsika} configuration utility (\texttt{coconut}).
    The compile time configuration of \gls{corsika} determines among other options the used
    interaction models. 
    The main purpose of storing the whole header file and not single options is to
    minimize the complexity and maintainability of \texttt{mopro3} by not implementing each possible option
    but rely on the header file generation through the \texttt{coconut} tool.
  \item[inputcard\_template, Text] The template for the input card  as discussed in \autoref{sec:mopro}
  \item[addtional\_files, Binary] On its first run, \gls{corsika} calculates a number
    of interpolation tables, which is quite time intensive and can take up to several hours.
    This is infeasible for a large, parallel production on a cluster.
    Thus, these interpolation tables are calculated once up front and stored as a tarball
    in the database, which is unpacked into the \gls{corsika} run directory for each
    job.
\end{description}


\subsection{CorsikaRun}
\begin{description}[font=\ttfamily, itemsep=0.2ex, parsep=0ex]
  \item[id, Integer, PrimaryKey] inserted as the run number into the input card template
  \item[corsika\_settings\_id, ForeignKey] reference to the \texttt{CorsikaSettings} to be used for this run.
  \item[primary\_particle, Integer] \gls{corsika} identification number of the primary particle,
    e.\,g.\ 1 for gamma ray, 14 for proton.
  \item[n\_showers, Integer] how many showers to generate in this run
  \item[zenith\_min, Float]
  \item[zenith\_max, Float] minimum and maximum zenith range
  \item[azimuth\_min, Float]
  \item[azimuth\_max, Float] minimum and maximum azimuth range
  \item[energy\_min, Float]
  \item[energy\_max, Float] minimum and maximum energy range
  \item[reuse, Integer] how many times the shower should be reused, see \autoref{sec:corsika-cherenkov}
  \item[max\_radius, Float] maximum scatter radius of the primary particle
  \item[bunch\_size] Bunch size option as discussed in \autoref{sec:corsika-cherenkov}
\end{description}

The \texttt{ceres} configuration is stored in the table \texttt{CeresSetttings}
and contains these columns:
\begin{description}[font=\ttfamily, itemsep=0.2ex, parsep=0ex]
  \item[id, Integer, PrimaryKey]
  \item[name, Text] human readable name, used for the directory structure and output files
  \item[revision, Integer] version number in \gls{fact}'s version control repository
  \item[rc\_template, Text] template for the \ceres{} rc file
  \item[resource\_files, Binary] A tarball with further configuration files.
    This includes for example tables of the reflectivity versus wavelength of the mirrors
    and the photon detection efficiency of the \glspl{sipm}.
  \item[psf\_sigma, Float] \gls{psf} of the individual mirror facets
  \item[apd\_dead\_time, Float]
  \item[apd\_recovery\_time, Float]
  \item[apd\_cross\_talk, Float]
  \item[apd\_afterpulse\_probability\_1, Float]
  \item[apd\_afterpulse\_probability\_2, Float] Physical properties
    of the \gls{sipm} cells, see \autoref{chp:fact}.
  \item[excess\_noise, Float] Standard deviation of white noise added to the electronics signal.
  \item[nsb\_rate, Float] Rate of \gls{nsb} photons
  \item[additional\_photon\_acceptance, Float] Photons are only processed with this probability,
    effectively discarding a fixed percentage of photons. 
    Introduced for \cite{phd-temme}, this was motivated by the missing obstruction
    by the telescope structure and degrading mirrors.
  \item[dark\_count\_rate, Float] of the \gls{sipm}
  \item[pulse\_shape\_function, Text] a function describing the shape of the single photon
    pulse
  \item[residual\_time\_spread, Float] Standard deviation of a fixed time offset between pixels
  \item[gapd\_time\_jitter, Float] Standard deviation for a normal random number added 
    to the arrival time of each individual Cherenkov photon
  \item[discriminator\_threshold] Trigger threshold, see \autoref{sec:camera}
\end{description}

\subsection{CeresRun}
\begin{description}[font=\ttfamily, itemsep=0.2ex, parsep=0ex]
  \item[id, Integer, PrimaryKey]
  \item[corsika\_run\_id, ForeignKey] id of the \texttt{CorsikaRun} used as input
  \item[ceres\_settings\_id, ForeignKey] id of the \texttt{CeresSettings} to be used for this run
  \item[diffuse, Boolean] if this run should simulate point-like or diffuse source of the primary.
    For the diffuse case, pointing direction of the telescope is drawn randomly around the direction of the primary particle.
  \item[off\_target\_distance, Float] wobble distance for point source mode or maximum angle
    to the optical axis for diffuse mode.
\end{description}

\subsection{Processing related fields for CorsikaRun and CeresRun}
Additionally, both \texttt{CorsikaRun} and \texttt{CeresRun} have these columns for
tracking job state
\begin{description}[font=\ttfamily, itemsep=0.2ex, parsep=0ex]
  \item[location, Text] Processing location, for the case of multiple consumers from
    the same database. Only \texttt{CeresRun}s are started where the corresponding 
    \texttt{CorsikaRun} was processed at the same location, so that the input file is available.
  \item[walltime, Integer] walltime is the maximum time a job is allowed to take and
    is required by most cluster job management tools like \texttt{SLURM} so that jobs can be scheduled.
  \item[status\_id, ForeignKey] id of the current job status 
  \item[priority, Integer] jobs are processed in order of \texttt{priority}, \texttt{id}.
    Lower values mean more important.
  \item[duration, Integer] The runtime of the job in seconds, filled after a run finished sucessfully
\end{description}

Also the path of the outputfiles, one for \texttt{CorsikaRun}, two for each \texttt{CeresRun}
are stored.

\section{CORSIKA Input Card Template}\label{apx:corsika_input}
\lstinputlisting{listings/corsika_template.tex}

\section{CERES rc template}\label{apx:ceres-template}
\lstinputlisting[basicstyle=\footnotesize\ttfamily]{listings/ceres_template.tex}


\section{Pedestal Variance}\label{apx:pedvar}
\begin{center}
  \includegraphics{build/plots/star_pedvar.pdf}
  \captionof{figure}{%
    Mean pedestal variance over 1000 events of a run observing Aldebaran,
    a variable star with an apparent magnitude of around \SI{0.8}{mag}.
    To calculate pedestal variance, the time series in each pixel are divided
    into windows of 30 values each.
    The values are summed in each window and the variance over all windows is then
    proportional to the amount of noise photons the pixel received.
    The gray circles show the positions of stars brighter than \SI{6}{mag} as
    queried through the star service.
    This also demonstrates nicely, that the conversion from celestial coordinates
    into camera coordinates works correctly, see \autoref{sec:coordinate-transforms}.
  }\label{fig:pedvar}
\end{center}



\section{FACT-Tools Standard Analysis XML}\label{apx:std_analysis}

\lstinputlisting[language=xml]{listings/facttools.xml}


\section{SQL Query for the Crab Dataset}\label{apx:crab-runs}
\lstinputlisting[language=sql]{listings/dataset.sql}

\section{Configuration for the \texttt{aict-tools}}\label{apx:aict-config}

\lstinputlisting{../configs/aict.yaml}

\newpage
\section{\param{size} for Excess Events and Gamma Simulations}\label{apx:gamma-size}
\begin{center}
  \includegraphics[scale=0.9]{build/plots/data_gamma_comp_apa95.pdf}
  \captionof{figure}{%
    Image parameter \param{size} for simulated gammas and the observed excess events
    in the on region for loose event selection criteria.
    The simulated dataset with $\param{APA} = \SI{95}{\percent}$ is shown here.
  }\label{fig:data-gamma-size-95}
\end{center}
\begin{center}
  \includegraphics[scale=0.9]{build/plots/data_gamma_comp_apa100.pdf}
  \captionof{figure}{%
    Image parameter \param{size} for simulated gammas and the observed excess events
    in the on region for loose event selection criteria.
    The simulated dataset with $\param{APA} = \SI{100}{\percent}$ is shown here.
  }\label{fig:data-gamma-size-100}
\end{center}

\section{Angular Resolution for all Events}\label{apx:old-disp-all}
\begin{center}
  \includegraphics[scale=0.85]{build/plots/compare_old_disp.pdf}
  \captionof{figure}{%
    Comparison of the angular resolution with the simple \param{disp} parameterization
    from \facttools{} with the new, machine learning based parameterization implemented
    in the \texttt{aict-tools} for the simulated point source gammas of the dataset
    with $\param{APA} = \SI{95}{\percent}$.
  }\label{fig:ang-comp-all}
\end{center}

\section{Bias and Resolution without Applying the Event Selection}\label{apx:bias-resolution-all}
\begin{center}
  \includegraphics[scale=0.85]{build/plots/bias_resolution_all_apa85.pdf}
  \captionof{figure}{%
    Bias and resolution for all gamma events simulated in wobble mode after
    applying only the event pre-selection.
  }\label{fig:bias-resolution-all}
\end{center}


\newpage
\section{Performance visualisations}\label{apx:apa-not-85}
All visualizations presented in \autoref{chp:performance} for
the datasets that were not shown there.

\begin{center}
  \includegraphics[scale=0.95]{build/plots/significances_apa95.pdf}
  \captionof{figure}{%
    Significances for \param{APA} = \SI{95}{\percent}, see \autoref{fig:significances}. 
  }\label{fig:significances-95}
\end{center}
\begin{center}
  \includegraphics[scale=0.95]{build/plots/significances_apa100.pdf}
  \captionof{figure}{%
    Significances for \param{APA} = \SI{100}{\percent}, see \autoref{fig:significances}. 
  }\label{fig:significances-100}
\end{center}


\begin{center}
  \includegraphics{build/plots/theta2_apa95.pdf}
  \captionof{figure}{%
    Theta-Squared-Plot for \param{APA} = \SI{95}{\percent}, see \autoref{fig:theta2}. 
  }\label{fig:theta2-95}
\end{center}
\begin{center}
  \centering
  \includegraphics{build/plots/theta2_apa100.pdf}
  \captionof{figure}{%
    Theta-Squared-Plot for \param{APA} = \SI{100}{\percent}, see \autoref{fig:theta2}. 
  }\label{fig:theta2-100}
\end{center}


\begin{center}
  \includegraphics{build/plots/effective_area_apa95.pdf}
  \captionof{figure}{%
    Effective area for \param{APA} = \SI{95}{\percent}, see \autoref{fig:effective-area}. 
  }\label{fig:aeff-95}
\end{center}
\begin{center}
  \includegraphics{build/plots/effective_area_apa100.pdf}
  \captionof{figure}{%
    Effective area for \param{APA} = \SI{100}{\percent}, see \autoref{fig:effective-area}. 
  }\label{fig:aeff-100}
\end{center}


\begin{center}
  \includegraphics{build/plots/bias_resolution_apa95.pdf}
  \captionof{figure}{%
    Bias and resolution for \param{APA} = \SI{95}{\percent}, see \autoref{fig:bias-resolution}. 
  }\label{fig:br-95}
\end{center}
\begin{center}
  \includegraphics{build/plots/bias_resolution_apa100.pdf}
  \captionof{figure}{%
    Bias and resolution for \param{APA} = \SI{100}{\percent}, see \autoref{fig:bias-resolution}. 
  }\label{fig:br-100}
\end{center}


\begin{center}
  \includegraphics{build/plots/sensitivity_apa95.pdf}
  \captionof{figure}{%
    Differential sensitivity for \param{APA} = \SI{95}{\percent}, see \autoref{fig:sensitivity}. 
  }\label{fig:sensitivity-95}
\end{center}
\begin{center}
  \includegraphics{build/plots/sensitivity_apa100.pdf}
  \captionof{figure}{%
    Differential sensitivity for \param{APA} = \SI{100}{\percent}, see \autoref{fig:sensitivity}. 
  }\label{fig:sensitivity-100}
\end{center}


\section{Estimated Energy vs. True Energy for Gammas and Protons}\label{apx:est-vs-true}
\begin{figure}
  \centering
  \includegraphics{build/plots/gamma_energy_prediction_vs_true_energy.pdf}
  \caption{%
    \Eest{}  vs. \Etrue{} for simulated proton and gamma ray induced showers.
    The contours contain the listed percentage of all events.
  }
\end{figure}


\section{Configuration for the Unfolding}\label{apx:unfolding-config}
\lstinputlisting{../configs/unfolding.yaml}
