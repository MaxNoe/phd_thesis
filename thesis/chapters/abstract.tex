\noindent\textbf{\sffamily\Large Abstract}\\[0.5\baselineskip]
High energy gamma-ray astronomy probes the most extreme phenomena in our universe:
super novae and their remnants as well as supermassive  black holes at the center
of far away galaxies.
The First G-APD Cherenkov Telescope (FACT) is a small, prototype Imaging Air Cherenkov Telescope (IACT)
operating since 2011 at the Roque de los Muchachos, La Palma, Spain. 
It specializes in continuously monitoring the brightest known sources of gamma rays.

In this thesis, I present a new, open analysis chain for the data recorded by FACT,
with a major focus on ensuring reproducibility and relying on modern,
well-tested tools with widespread adoption.

The integral sensitivity of FACT was improved by \SI{45}{\percent} compared to previous analyses
by the introduction of an improved algorithm for the reconstruction of
origin of the gamma rays and many smaller improvements in the preprocessing.
Sensitivity is evaluated both on simulated datasets as well as observations of the Crab Nebula,
the \enquote{standard candle} of gamma-ray astronomy.  

Another major advantage of this new analysis chain is the elimination of
the dependence on a known point source position from the event reconstruction,
thus enabling the creation of skymaps, the analysis of observations where the source position is not exactly known
and sharing reconstructed events in the now standardized format for open gamma-ray astronomy.
This has lead to the first publication of a joined, multi-instrument analysis on open data
of four currently operating Cherenkov telescopes.

A smaller second part of this thesis is concerned with enabling 
robotic operation of FACT, which is now the first Cherenkov telescope,
where no operators are required during regular observations.

\bigskip
\begin{otherlanguage}{ngerman}
\noindent\textbf{\sffamily\Large Zusammenfassung}\\[0.5\baselineskip]
Die Hochenergie-Gammaastronomie erlaubt es, die extremsten Phänomene in unserem Universum
zu untersuchen: Supernovae und ihre Überreste sowie supermassive schwarze Löcher in den
Zentren weit entfernter Galaxien.
  Das First G-APD Cherenkov Telescope (FACT) ist ein kleines, bildgebendes, atmosphärisches Tscherenkow Teleskop, dass seit Oktober 2011 auf dem Roque de los Muchachos, La Palma, Spanien beobachtet.
Es ist auf die Langzeitbeobachtung der hellsten bekannten Gammastrahlungsquellen spezialisiert.

In dieser Arbeit stelle ich eine neue, öffentliche Analysekette für die von FACT aufgenommen
Daten vor.
Ein Hauptaugenmerk wurde auf die Reproduzierbarkeit und die Verwendung moderner,
gut getesteter und weit verbreiteter Methoden gelegt.
Die integrale Sensitivität von FACT wurde im Vergleich zu früheren Analysen um \SI{45}{\percent}
gesteigert, hauptsächlich durch die Einführung einer verbesserten Methode zur Bestimmung der
Herkunft der Gammastrahlung, sowie durch viele weitere, kleinere Verbesserungen in der Vorverarbeitung.
Die Sensitivität wurde sowohl auf simulierten Daten als auch auf Beobachtungen des Krebsnebels,
der Standardkerze der Gammaastronomie, ausgewertet.

Ein weiterer Vorteil der neuentwickelten Analysekette ist ihre Unabhängigkeit von
Annahmen über eine bekannte Punktquelle.
Dies ermöglicht die Erstellung von Himmelskarten, die Analyse von Beobachtungen,
bei denen die Quellposition nicht genau bekannt ist und das Speichern und Veröffentlichen
rekonstruierter Ereignisse im nun standardisiertem Datenformat für offene Gammaastronomie.
Die hat die Publikation der ersten gemeinsamen Analyse von Krebsnebel-Daten von vier 
aktuell beobachtenden Tscherenkow-Teleskopen ermöglicht.

Der zweite, kleinere Teil dieser Arbeit beschäftigt sich mit der Robotisierung von FACT,
welches nun das erste Tscherenkow-Teleskop ist, für dessen reguläre Observationen kein Personal
mehr benötigt wird.


\end{otherlanguage}
